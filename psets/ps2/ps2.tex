\documentclass[12pt]{amsart}

%Below are necessary packages for your course.
\usepackage{amsfonts,latexsym,amsthm,amssymb,amsmath,amscd,euscript}
\usepackage{mathrsfs,manfnt,enumitem,stmaryrd}
\usepackage[super]{nth}
\usepackage{setspace}
\usepackage{titlesec}
\usepackage{framed}
\usepackage{verbatim}
\usepackage{calc,color,relsize}
\usepackage{amsxtra}
%\usepackage{eucal}
\usepackage[all]{xy}
\usepackage{graphicx,mathabx}
\usepackage[geometry]{ifsym}
\usepackage{hyperref}
    \hypersetup{colorlinks=true,citecolor=blue,urlcolor =black,linkbordercolor={1 0 0}}
\usepackage{mathtools}
%\usepackage{fc}
\usepackage{pifont}
\usepackage{tikz}
\usepackage{tikz-cd}
\usetikzlibrary{matrix,arrows,decorations.pathmorphing}
\textwidth=14.5cm \textheight=20.5cm
 \oddsidemargin=1cm
\evensidemargin=1cm

\allowdisplaybreaks[1]

%Below are the theorem, definition, example, lemma, etc. body types.

\newtheorem{theorem}{Theorem}
\newtheorem*{proposition}{Proposition}
\newtheorem{lemma}[theorem]{Lemma}
\newtheorem{corollary}[theorem]{Corollary}
\newtheorem{conjecture}[theorem]{Conjecture}
\newtheorem{postulate}[theorem]{Postulate}
\theoremstyle{definition}
\newtheorem{defn}[theorem]{Definition}
\newtheorem{example}[theorem]{Example}

\theoremstyle{remark}
\newtheorem*{remark}{Remark}
\newtheorem*{notation}{Notation}
\newtheorem*{note}{Note}

%This is a list of new commands I defined to make your life easier. Use them as the occur in the course.
\newcommand{\sub}{\operatorname{sub}}
\newcommand{\quot}{\operatorname{quot}}
\newcommand{\bw}{\bigwedge}
\newcommand{\Avs}{\operatorname{Av}^{\operatorname{sign}}}
\newcommand{\bad}{\operatorname{bad}}
\newcommand{\sign}{\operatorname{sign}}
\newcommand{\id}{\operatorname{id}}
\newcommand{\defeq}{\vcentcolon=}
\newcommand{\eqdef}{=\vcentcolon}
%We can even define a new command for \newcommand!
\newcommand\nc{\newcommand}
\nc{\on}{\operatorname}
\nc\renc{\renewcommand}
\nc{\BR}{\mathbb R}
\nc{\BC}{\mathbb C}
\nc{\BQ}{\mathbb Q}
\nc{\BZ}{\mathbb Z}
\nc{\BN}{\mathbb N}
\nc{\BS}{\mathbb S}
\nc{\Hom}{\on{Hom}}
\nc{\wt}{\widetilde}
\nc{\vspan}{\on{span}}
\nc{\ord}{\on{ord}}
\nc{\im}{\on{im}}
\nc{\Mat}{\on{Mat}}
\nc{\can}{\on{can}}
\nc{\coker}{\on{coker}}
\nc{\ev}{\on{ev}}
\nc{\Tr}{\on{Tr}}
\nc{\End}{\on{End}}
\nc{\swap}{\on{swap}}
\nc{\Set}{\on{Set}}
\nc{\bC}{{\mathbf C}}
\nc{\bc}{{\mathbf c}}
\nc{\bD}{{\mathbf D}}
\nc{\bd}{{\mathbf d}}
\nc{\bE}{{\mathbf E}}
\nc{\be}{{\mathbf e}}
\nc{\bF}{{\mathbf F}}
\nc{\bff}{{\mathbf f}}
\nc{\CE}{\mathcal E}
\renc{\mod}{\on{-mod}} %Careful - turn this off in a number theory setting
\newcommand{\spec}{\text{spec}}
\nc{\adj}{\on{adj}}
\nc{\tensor}[3]{#1 \underset{#2}\otimes #3}
\nc{\Nat}{\on{Nat}}
\nc{\op}{\on{op}}
\nc{\Funct}{\on{Funct}}
\nc{\Ob}{\on{Ob}}
\nc{\fR}{\mathfrak{R}}
\nc{\Vect}{\on{Vect}}
\nc{\ns}{\on{non-spec}}
\nc{\ol}{\overline}
\nc{\ul}{\underline}
\nc{\univ}{\on{univ}}
\nc{\Maps}{\on{Maps}}
\nc{\bdd}{\on{bdd}}
\nc{\cont}{\on{cont}}
\nc{\Sym}{\on{Sym}}
\nc{\vol}{\on{vol}}
\nc{\supp}{\on{supp}}
\nc{\Lie}{\on{Lie}}
\nc{\master}{\on{master}}
\nc{\pt}{\on{pt}}

\title{M11482 Problem Set $2$}%Change the PSet number as needed.
\date{\today}


\begin{document}

\maketitle

\vspace*{-0.25in}
\centerline{Leo Alcock, Luis Alvernaz}
\centerline{\href{mailto:M11482-teachers@esp.mit.edu}{{\tt M1182-teachers@esp.mit.edu}}}
\vspace*{0.15in}
\centerline{{\it Problems marked with * are more difficult; we think...}}
\vspace*{0.25in}

\medskip

\noindent{{\bf problem numba} Problem statement}


\medskip

\noindent{{\bf 1: Open balls are open} Given a metric space $(X,d)$, we defined
the open ball, $N_r(p)$ about a point $p\in X$ of radius $r$ to be $$N_r(p) = \{ q\in X |\: d(p,q) < r\}$$ We defined a set $S\subset X$ to be open if for every element in $S$ an open ball around that point is contained in $S$. Show that open
balls are open and conclude after that we can equivalently define a set $S$ to be
open if it is the union of a collection of open balls.

}

\medskip

\noindent{{\bf 2: Open/Closed relationship} We defined a closed set to be a subset which contains all of it's limit points. Show that a set $S\subset X$ is closed
if and only if the complement of it (in $X$) is open (and vice versa). }

\medskip

\noindent{{\bf 3: Topology?} Show that an arbitrary union of open sets $\{U_\alpha\}$ is open in $X$ and that finite intersections of open sets are open. Conclude by this and problem 2 that arbitrary intersections of closed sets are closed and finite unions of closed sets are closed.\footnote{Good exercise to find an arbitrary intersection of open sets which isn't open and and an arbitrary union of closed sets which isn't closed}}

\medskip

\noindent{{\bf 4: Circle, diamond, and square metric} With $X = \BR^2$, recall the following three metrics defined in class:
\begin{align*}
d_1(p,q) &:= \sqrt{(x_1 - x_2)^2 + (y_1 - y_2)^2}\\
d_2(p,q) &:= |x_1 - x_2| + |y_1 - y_2|\\
d_3(p,q) &:= \max (|x_1 - x_2|, |y_1 - y_2|)
\end{align*}

With these three metrics, we get three different metric spaces: $(X,d_1)$, $(X,d_2)$, and $(X,d_3)$. Show that a set is open in one of these spaces if and only if it is open in another. \textit{hint: Show given $N_{r,d_i}(p)$ there exists $r' > 0$ such that $N_{r',d_j}(p)\subset N_{r,d_i}(p)$.}


}

\medskip

\noindent{{\bf 5*: Closure} Given an arbitrary subset $S\subset X$ of some metric space $(X,d)$, define $\bar{S} = \displaystyle \bigcap_{C\in \mathscr{C}_S} C$ where $\mathscr{C}_S$ is the collection of all closed sets containing $S$. Show that $\bar{S}$ is the union of $S$ and all of $S$'s limit points. 

\noindent Can you define a corresponding notion to closure for open sets?
}

\medskip

\noindent{{\bf 6*: 2 Boxes} We define the following two boxes:
\begin{align*}
B_1 &= \{(x,y)\in \BR^2 |\: 0\le x,y \le 1 \}\\
B_2 &= \{(x,y)\in \BR^2 |\: 2\le x,y\le 3\}
\end{align*}
We let $B = B_1 \bigcup B_2$ and consider it as a metric subspace or $\BR^2$ (under any of the metrics $d_i$ mentioned in problem 4). What are the subsets of $B$ which are closed and open (sometimes called "clopen")? (no need/attempt for proof on this... didn't introduce enough things to prove the answer for this question).
}
\end{document}






























