\documentclass[12pt]{amsart}

%Below are necessary packages for your course.
\usepackage{amsfonts,latexsym,amsthm,amssymb,amsmath,amscd,euscript}
\usepackage{mathrsfs,manfnt,enumitem,stmaryrd}
\usepackage[super]{nth}
\usepackage{setspace}
\usepackage{titlesec}
\usepackage{framed}
\usepackage{verbatim}
\usepackage{calc,color,relsize}
\usepackage{amsxtra}
%\usepackage{eucal}
\usepackage[all]{xy}
\usepackage{graphicx,mathabx}
\usepackage[geometry]{ifsym}
\usepackage{hyperref}
    \hypersetup{colorlinks=true,citecolor=blue,urlcolor =black,linkbordercolor={1 0 0}}
\usepackage{mathtools}
%\usepackage{fc}
\usepackage{pifont}
\usepackage{tikz}
\usepackage{tikz-cd}
\usetikzlibrary{matrix,arrows,decorations.pathmorphing}
\textwidth=14.5cm \textheight=20.5cm
 \oddsidemargin=1cm
\evensidemargin=1cm

\allowdisplaybreaks[1]

%Below are the theorem, definition, example, lemma, etc. body types.

\newtheorem{theorem}{Theorem}
\newtheorem*{proposition}{Proposition}
\newtheorem{lemma}[theorem]{Lemma}
\newtheorem{corollary}[theorem]{Corollary}
\newtheorem{conjecture}[theorem]{Conjecture}
\newtheorem{postulate}[theorem]{Postulate}
\theoremstyle{definition}
\newtheorem{defn}[theorem]{Definition}
\newtheorem{example}[theorem]{Example}

\theoremstyle{remark}
\newtheorem*{remark}{Remark}
\newtheorem*{notation}{Notation}
\newtheorem*{note}{Note}

%This is a list of new commands I defined to make your life easier. Use them as the occur in the course.
\newcommand{\sub}{\operatorname{sub}}
\newcommand{\quot}{\operatorname{quot}}
\newcommand{\bw}{\bigwedge}
\newcommand{\Avs}{\operatorname{Av}^{\operatorname{sign}}}
\newcommand{\bad}{\operatorname{bad}}
\newcommand{\sign}{\operatorname{sign}}
\newcommand{\id}{\operatorname{id}}
\newcommand{\defeq}{\vcentcolon=}
\newcommand{\eqdef}{=\vcentcolon}
%We can even define a new command for \newcommand!
\newcommand\nc{\newcommand}
\nc{\on}{\operatorname}
\nc\renc{\renewcommand}
\nc{\BR}{\mathbb R}
\nc{\BC}{\mathbb C}
\nc{\BQ}{\mathbb Q}
\nc{\BZ}{\mathbb Z}
\nc{\BN}{\mathbb N}
\nc{\BS}{\mathbb S}
\nc{\Hom}{\on{Hom}}
\nc{\wt}{\widetilde}
\nc{\vspan}{\on{span}}
\nc{\ord}{\on{ord}}
\nc{\im}{\on{im}}
\nc{\Mat}{\on{Mat}}
\nc{\can}{\on{can}}
\nc{\coker}{\on{coker}}
\nc{\ev}{\on{ev}}
\nc{\Tr}{\on{Tr}}
\nc{\End}{\on{End}}
\nc{\swap}{\on{swap}}
\nc{\Set}{\on{Set}}
\nc{\bC}{{\mathbf C}}
\nc{\bc}{{\mathbf c}}
\nc{\bD}{{\mathbf D}}
\nc{\bd}{{\mathbf d}}
\nc{\bE}{{\mathbf E}}
\nc{\be}{{\mathbf e}}
\nc{\bF}{{\mathbf F}}
\nc{\bff}{{\mathbf f}}
\nc{\CE}{\mathcal E}
\renc{\mod}{\on{-mod}} %Careful - turn this off in a number theory setting
\newcommand{\spec}{\text{spec}}
\nc{\adj}{\on{adj}}
\nc{\tensor}[3]{#1 \underset{#2}\otimes #3}
\nc{\Nat}{\on{Nat}}
\nc{\op}{\on{op}}
\nc{\Funct}{\on{Funct}}
\nc{\Ob}{\on{Ob}}
\nc{\fR}{\mathfrak{R}}
\nc{\Vect}{\on{Vect}}
\nc{\ns}{\on{non-spec}}
\nc{\ol}{\overline}
\nc{\ul}{\underline}
\nc{\univ}{\on{univ}}
\nc{\Maps}{\on{Maps}}
\nc{\bdd}{\on{bdd}}
\nc{\cont}{\on{cont}}
\nc{\Sym}{\on{Sym}}
\nc{\vol}{\on{vol}}
\nc{\supp}{\on{supp}}
\nc{\Lie}{\on{Lie}}
\nc{\master}{\on{master}}
\nc{\pt}{\on{pt}}

\title{M11482, Problem Set $1$}%Change the PSet number as needed.
\date{\today}


\begin{document}

\maketitle

\vspace*{-0.25in}
\centerline{Leo Alcock, Luis Alvernaz}
\centerline{\href{mailto:M11482-teachers@esp.mit.edu}{{\tt M1182-teachers@esp.mit.edu}}}
\vspace*{0.15in}
\centerline{{\it Problems marked with * are more difficult; we think...}}
\vspace*{0.25in}

Do between 0 and 5 of the following problems and we'll correct your solutions. Cheers ~



\medskip

\noindent{{\bf 1} Define the order relation $<$ on $\BZ$ using only the addition structure(s)\footnote{note that this includes subtraction} on $\BZ$}


\medskip

\noindent{{\bf 2*} Given a (totally) ordered set $(S,\preceq)$\footnote{this denotes a set $S$ with ordering given by $\preceq$}, we may also define the greatest lower bound property as follows:

\begin{defn}
Given an ordered set $(S,\preceq)$, this set has the \textbf{greatest lower bound property} if for all nonempty
subsets $E\subset S$ with a lower bound (some $\beta \in S$ such that $\forall e\in E,\: \beta\le e$) $E$ has a \textit{greatest lower bound}, $\beta^*$, such that for all lower bounds $\beta$ of $E$, $\beta \le \beta^*$.
\end{defn}

Prove that if an ordered set has the least upper bound property it also has the greatest lower bound property.

If an ordered set has the greatest lower bound property does it necessarily have the least upper bound property?
}

\medskip

\noindent{{\bf 3} Given $a,b \in \BC, a\neq b$, form the set 
$$S = \{z\in \BC| |z-a| = |z-b|\}$$

What does this set look like?
}

\medskip

\noindent{{\bf 4} We can make $\BC$ an ordered set by giving it the \textit{dictionary ordering} inherited from $\BR$: 
$(x,y), (u,v)\in \BC,\: (x,y) < (u,v)$ if $x < u$ or $x = u$ and $y < v$. Prove that $\BC$ is not an ordered field under the \textit{dictionary ordering}.
}

\medskip 

\noindent{{\bf 5**} Prove that there is no ordering you can put on $\BC$ which makes it an ordered field. (\textit{Hint: Start with proving that in any ordered field, $-1 < 0 < 1$.})}












\end{document}